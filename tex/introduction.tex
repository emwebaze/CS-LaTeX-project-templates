\chapter{Introduction}

\section{Background}
Cassava Mosaic Disease is majorly spread by the whiteflies which move from leaf to leaf in different cassava plantations. The whiteflies, comprising only the family Aleyrodidae, are small hemipterans and more than 1550 species have been described. Whiteflies typically feed on the underside of plant leaves. The ability of the whitefly to carry and spread disease is the widest impact they have had on global food production. 

In the tropics and subtropics, whiteflies (Hemiptera: Aleyrodidae) have become one of the most serious crop protection problems. Economic losses are estimated in the hundreds of millions of dollars. While several species of whitefly cause crop losses through direct feeding, a species complex, or group of whiteflies in the genus Bemisia are important in the transmission of plant diseases. Bemisia tabaci and B. argentifolii, transmit African cassava mosaic.

\subsection{Disease detection in Cassava}

However, there is continuous need for timely and accurate information pertaining to their spread from region to region for proper management of the CMD incidence and severity. Information about the infestation rates should be got. This information would be used in monitoring and forecasting the spread of CMD over time and planning appropriate interventions to avert crises. 

Conversely, such information is difficult to obtain at present, due to challenges such as the unavailability of suitable technical staff with the expertise to detect the white flies, the time and cost incurred by transport to rural regions of the country, unavailability of salaries for the field staff, and impassable roads during rainy seasons in some regions of the country, and the time taken to coordinate paper reports.

We propose a technique that facilitates automated whitefly count based on computer vision.

\section{Problem Statement}
Agriculturalists at Namulonge Research Center have over the years adopted the traditional manual counting of whiteflies which solely depends on observer’s skill. The whole process is very tedious considering that the whiteflies are very small and volatile so it is possible to lose track of the numbers.  In addition, a single cassava leaf usually has very many whiteflies on it which increases the difficulty in counting and consumes a lot of time. The fact that these flies reside on the back side of the cassava leaves makes the process all the more complicated because in the event of turning the leaf, the flies keep flying away which is really frustrating. 
The researchers handle large volumes of data from many sources and in such a case, data should be easy to collect manage and used to curb the spread of cassava mosaic disease but this has not been easy with the manual method.
Coming up with a technique to automatically identify and count the whiteflies based on computer vision will be of great benefit to both the researchers and agriculturalists because it will ease and fasten their tasks.
\section{Main Objective}
The major objective of the study was to develop a real time mobile application that automates the process of detection and counting of whiteflies on cassava leaves using computer vision techniques. 
\subsection{Specific Objectives}
The specific objectives of the study were: 
\begin{itemize}
\item[i.] To carry out a maiden study on the literature of existing computer vision techniques used in object detection.
\item[ii.] To carry out a preliminary study on the existing system used to identify and count whiteflies by agricultural researchers.
\item[iii.] To design the system for the automated whitefly count.
\item[iv.] To implement the designed system.
\item[v.] Finally, to carry out the testing process of the implemented system.
\end{itemize}

\section{Scope of the study}
This project limited its self to a mechanism by which farmers place their basic camera phones over the leaf infested with whiteflies, the images are then diagnosed real time for the presence of white flies and a count is provided.  The study is limited to the automation of whitefly count as it is part of a bigger project aimed at classifying the levels of cassava mosaic infection on cassava plants.

\section{Significance of the study}
The proposed system will provide the following advantages:
Provide an efficient way of counting and tracking the spread of the white flies in the country at large.
The project will also help the farmers and researchers to know the different prevalent infestation rates of the white flies in particular regions.
The project once completed will also reduce on the expenses incurred by the research center especially in facilitating various field staff that are deployed in various regions to try and track the spread of the white flies.
It will also provide room for further study in this area as well as provide reference literature for future scholars intending to build on this area of research.